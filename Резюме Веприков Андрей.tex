% "Станет проще"

\documentclass[a4paper,12pt]{article} % тип документа

% report, book

%  Русский язык

\usepackage[T2A]{fontenc}			% кодировка
\usepackage[utf8]{inputenc}			% кодировка исходного текста
\usepackage[english,russian]{babel}	% локализация и переносы


% Математика
\usepackage{amsmath,amsfonts,amssymb,amsthm,mathtools} 
\usepackage[left=20mm, top=20mm, right=20mm, bottom=20mm, nohead, nofoot]{geometry}

% Картинки
\usepackage{graphicx}
\graphicspath{{pictures/}}
\DeclareGraphicsExtensions{.pdf,.png,.jpg}

% подключаем hyperref (для ссылок внутри  pdf)
\usepackage[unicode, pdftex]{hyperref}

\makeatletter
\renewcommand{\@maketitle}{
\center
\LARGE \textbf{\@title}
\par\text{\@author}
\par}
\makeatother

\usepackage{wasysym}
%%\documentclass[12pt]{article}
\usepackage{tikz}

%Заговолок
\author{Веприков Андрей Сергеевич}
\title{Резюме}

\begin{document} % начало документа

\maketitle

\section*{Личные данные}

\begin{enumerate}

\item[$\bullet$] Дата рождения: 21.12.2001

\item[$\bullet$] Телефон: 89127596252

\item[$\bullet$] Почта: veprikov.as@phystech.edu

\item[$\bullet$] Ссылка на GitHub: \url{https://github.com/Vepricov} 

\end{enumerate}

\section*{Образование}

\begin{enumerate}

\item[$\bullet$] Школа: МБОУ ИЕГЛ <<Школа №30>> города Ижевска

\item[$\bullet$] ВУЗ: МФТИ, Физтех-школа прикладной математики и информатики, направление <<прикладная математика и физика>>

Окончил первый курс, сейчас учусь на втором. Средний балл за 3 семестра равен 8.54 (по 10-ти бальной шкале)

\end{enumerate}

\section*{Навыки}

\begin{enumerate}

\item[$\bullet$] Самостоятельно изучал в школе язык программирования C++

\item[$\bullet$] Весь первый курс мы изучали язык программирования Си

\item[$\bullet$] Проходил курсы на курсере по Python 2 и Python 3

\item[$\bullet$] В 1 семестре изучал язык SQL

\item[$\bullet$] Знание различных алгоритмов программирования и структур данных

\item[$\bullet$] Знание математической статистики и теории вероятностей

\end{enumerate}

\section*{Дополнительные курсы}

\begin{enumerate}

\item[$\bullet$] Coursera специализация <<Машинное обучение и анализ данных>> (Python 2)

\item[$\bullet$] Coursera курс <<Python для анализа данных>> (Python 3)

\item[$\bullet$] 6 раз ездил в математические лагеря ЛОШ и ЗОШ при МФТИ, где изучал олимпиадную математику, получал там оценки 4 и 5 (по 5-ти бальной шкале) в сильных группах.

\end{enumerate}

\end{document}